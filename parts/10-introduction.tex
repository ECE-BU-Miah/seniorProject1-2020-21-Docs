%======================================================================
\chapter{Introduction}
\label{ch: Chapter1}
%======================================================================

\section{Project Description}
The Internet of Things is a large network of embedded devices such as sensors, wearables, and appliances capable of receiving control commands and reporting
data over the Internet. Many industries take advantage of this infrastructure to
improve process flow and efficiency. One of these industries is the commercial
and industrial sector which often employ building automation or management
systems to help better manage processes or devices in a building like air
conditioning systems, lights or industrial machines. Technological advancements
have allowed devices to become connected to a building network through
communication protocols like WiFi, Ethernet, Modbus, or BACnet which enables
them to be more easily controlled from a central system.


Our overall goal is to create a platform in which users can login and access devices
connected to a building's energy supply. This will allow the user to closely
monitor energy usage throughout a commercial or residential building.
Additionally, the user will be able to control devices connected to the
platform, allowing precise control over the building's energy use. The minimum
viable product of this project will be to connect to 1 to 3 devices via the the
platform and control them.

\section{Prior Work}
The previous senior project group worked on creating a remote motor control system with a Raspberry Pi and XBee module. To supply power directly to the motor and XBee with a single power supply, they used a buck-boost converter capable of stepping down power supply voltage to 3.3 V. They implemented a Python script that uses the Linux command \texttt{nmap} to search for the Raspberry Pi on the network. The Raspberry Pi connects to the transmitter XBee via a serial connection to transmit commands to the receiver XBee mounted to the motor. They were able to simply toggle the motor on and off and partially implemented this functionality in BEMOSS (Building Energy Management Open Source Software). A second portion of their project was implementing an HVAC control algorithm capable of controlling the temperature of a home with the Linear Quadratic Regulator control algorithm.

During a research project led by Miah \textit{et al.,} the DC motor interface was successfully
implemented into the BEMOSS platform which required adding a great deal of source code
to the platform to fully implement the device as BEMOSS does not support adding
new devices currently. The research also consisted of adding new
features such as a logo to the
platform for the motor. A possible improvement our platform will make is helping
developers on the project add new devices more easily.

One of the improvements our project makes over the previous is the use of UDP (user datagram protocol) multicasting which does not require any superuser privileges to locate an
embedded computer (Beaglebone Blue, for example). This implementation is also more convenient
because it does not require installing any packages as multicasting is natively
supported by most embedded Linux systems. If in the future, multiple embedded
computers are needed to be discovered, the multicasting system works elegantly
as each device can simply listen on a provided multicast group which consists of
an IP address and port number.

\section{Background Study}
A great deal of research has been conducted in the field of building energy management. Multiple different platforms have been developed that demonstrate the vast amount of features and possibilities that exist in developing these types of software.

One such platform is BEMOSS from the Virginia Polytechnic Institute and State
University Advanced Research Institute. BEMOSS features include being
open-source, allowing for multiple communication technologies, and supporting
many common IoT devices. This system allows for monitoring and controlling a
variety of devices securely~\cite{BEMOSS}.

Another open-source platform is rEMpy or residential Energy Management in
Python. The main focus of this platform developed by a team of researchers at
Universit\`{a} Politecnica delle Marche in Ancona, Italy was to simulate the
energy flow of a residential home. With the data collected from simulations, it
was possible for them to use forecasting and prediction algorithms to predict
various quantities such as energy usage and power consumption over a given time period. Their energy management system consists of a well-defined structure with components like a user interface, database, and optimal scheduler that communicate with modules like a prediction and thermal model. In terms of software technologies, it uses the Django web framework which provides a lot of built-in functionality for web development like a user-authentication system and object-relational mapping. Adapting the project to this framework could be a great way to help improve security and optimize database queries~\cite{fagiani2017}.

One of the works in the literature discusses Model Predictive Control (MPC)
which is a modern process control algorithm capable of taking into account
current and previous time values to improve building
automation~\cite{Mayer2017}. A multilevel hierarchy is presented to split
control of a building into two levels: the energy supply level and user level.
Example usage of this control scheme provided are temperature control of a
building and its corresponding zones and interaction with a smart grid. Studies
were performed on a commercial building with the building's energy supply system
and hierarchy model predictive controller implemented in Matlab.

In~\cite{8246800}, authors describe similar functional requirements to our
own BEMS Core. They periodically gather power consumption and device status data
and send that data to a database. We are doing exactly what they describe using
the Apache Cassandra database. The authors also use an \say{analytics [sic] engine
to process [the data] and generate reports, graphs, and charts.} Our system only
generates a power consumption plot for a given day, but we are still meeting
this same requirement - just on a smaller scale. The communication with devices
and power consumption reports are to be accessible through a web-based
application that is easy to use. Again, our implementation is simple and very
intuitive for the end-user. Finally, these authors mention a user-login feature.
They explain that their system would include multiple levels of accessibility
based on the user privileges. 

A control scheme in~\cite{Barchi2018} is presented to manage a photovoltaic
array and battery energy storage system. Tests were conducted in a shopping mall
with an electronic load to emulate the power consumption of the building.
Potentially, a similar technique could be used to model the energy demand of a
house in Simulink or Simscape model. Through their scheme, an intelligent BEMS
is used to collect measurable data such as power, voltage, and current and stored for offline analysis later. Their platform collects data every five minutes compared to the 20 second polling rate currently configured in our platform. A control algorithm was presented proving that grid energy is consumed more heavily during periods of low power demand rather than PV power.

% \section{Problem Statement}
% Despite several open-source building energy management platforms, such as BEMOSS, available in the market, the need for an effective and efficient open-source building energy management platform that takes into account the planning and technical challenges in integrating different load controllers
% with the microgrid as well as the smart grids is quite prominent. The proposed energy management platform is expected to address the huge energy demand in residential and commercial buildings despite limited number of sources of energy by incorporating learning, control, and estimation strategies as discussed in the following section.

% A high level architecture of the proposed building energy management system integrating different components of power systems and smart devices is shown in Figure 2. A microgrid is connected to the main grid through a Point of Common Coupling (PCC) to enable or disable islanded mode operation. Various renewable DERs are connected within the microgrid which have the capability of either supplying energy directly to the main grid or storing energy in energy storage devices such as batteries or supercapacitors. The core system will facilitate communication between many different building systems and electrical devices within a building. Examples of different systems include lighting systems and HVAC systems. Smart meters can enable two-way communication between an energy supplier and the BEMS. Various IoT appliances and devices in the building including washing machines, refrigerators, smart TVs, dishwashers, and WiFi-enabled PHEV chargers can be configured for communication with the central system to help optimize electrical consumption. Uncontrollable loads like computers and microwaves may be monitored for power consumption statistics and temperature measuring devices can be connected for determining proper HVAC operation. Utility operators, Ameren, for instance, can be utilized for demand response purposes. Lastly, mobile devices including smart phones and tablets can be used to remotely monitor data from the BEMS core. Usually, a building energy management system is primarily aimed in minimizing energy consumption through building automation strategies and the usage of energy management policies
% and algorithms. The building energy management platform, BEMOSS, developed by researchers at the Virginia Tech research lab, focuses mainly on incorporating building automation strategies, which is currently being used as a starting point for the proposed BEMS platform. The existing BEMOSS platform, for example, contains features for controlling electronic/electrical devices such as scheduling, power consumption monitoring, device control, and usage monitoring. However, no learning, control, and estimation strategies have been developed to help buildings lower energy costs, address the energy demand and smart integration microgrid and smart grid, which proves to be a potential shortcoming of the current BEMOSS.

The ultimate goal of this research is to design and develop a new building energy management
platform incorporating learning, control, and estimation strategies, which is expected to remedy the
shortcomings of some existing platforms, BEMOSS, for instance.

%----------------------------------------------------------------------
\section{Report Organization}
%----------------------------------------------------------------------
This report is organised into 5 chapters.  Chapter \ref{ch: Chapter1} discusses
what this project hopes to accomplish and what similar projects have completed.
Chapter \ref{ch: Chapter2} explains the architecture. Chapter \ref{ch: Chapter3}
provides a brief explanation of the control algorithms and architectures used.
Chapter \ref{ch: Chapter4} contains the results from a few experiments. Chapter \ref{ch: Chapter5} concludes the report and provides insight into future directions.

%%% Local Variables:
%%% mode: latex
%%% TeX-master: "../finalReport"
%%% End:
