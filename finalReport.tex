% uOttawa (unofficial) Thesis Template for LaTeX 
% Edited by Wail Gueaieb based on Stephen Carr's uWaterloo Template

% The files included in this package are slighly modified by Suruz Miah to adapt partial requirements  in writing project/thesis reports of the Bradley University's Department of Electrical and Computer Engineering.

% DON'T USE THIS TEMPLATE IF YOU DON'T KNOW WHAT YOU'RE DOING!
% Remember, it comes WITH NO WARRANTY!

% Please read the "00readme.txt" file first.
% Here is how to use this template:
%
% DON'T FORGET TO ADD YOUR OWN NAME AND TITLE in the "hyperref" package
% configuration in the "thesis-preample.tex" file. THIS INFORMATION GETS 
% EMBEDDED IN THE PDF FINAL PDF DOCUMENT.
% You can view the information if you view Properties of the PDF document.

% The template is based on the standard "book" document class which provides 
% all necessary sectioning structures and allows multi-part theses.

% DISCLAIMER
% To the best of our knowledge, this template satisfies the current 
% uOttawa thesis requirements.
% However, it is your responsibility to assure that you have met all 
% requirements of the university and your particular department.
% Many thanks to the feedback from many graduates that assisted the 
% development of this template.

% -----------------------------------------------------------------------

% When using pdflatex, by default the output is geared toward generating a PDF 
% version optimized for viewing on an electronic display, including 
% hyperlinks within the PDF.
 
% E.g. to process a thesis based on this template, run:

% (pdf)latex thesisMain	-- first pass of the (pdf)latex processor
% bibtex thesisMain 	-- generates bibliography from .bib data file(s) 
% (pdf)latex thesisMain	-- fixes cross-references, bibliographic references, etc
% (pdf)latex thesisMain	-- fixes cross-references, bibliographic references, etc
% makeindex -s nomentbl.ist -o thesisMain.nls thesisMain.nlo
% (pdf)latex thesisMain	-- fixes cross-references, bibliographic references, etc
% (pdf)latex thesisMain	-- fixes cross-references, bibliographic references, etc



% N.B. The "pdftex" program allows graphics in the following formats to be
% included with the "\includegraphics" command: PNG, PDF, JPEG, TIFF
% Tip 1: Generate your figures and photos in the size you want them to appear
% in your thesis, rather than scaling them with \includegraphics options.
% Tip 2: Any drawings you do should be in scalable vector graphic formats:
% SVG, PNG, WMF, EPS and then converted to PNG or PDF, so they are scalable in
% the final PDF as well.
% Tip 3: Photographs should be cropped and compressed so as not to be too large.

% To create a PDF output that is optimized for double-sided printing: 
%
% 1) comment-out the \documentclass statement in the preamble below, and
% un-comment the second \documentclass line.
%
% 2) change the value assigned below to the boolean variable
% "PrintVersion" from "false" to "true".

% --------------------- Start of Document Preamble -----------------------

% Specify the document class, default style attributes, and page dimensions
% For hyperlinked PDF, suitable for viewing on a computer, use this:
% \UseRawInputEncoding

% \documentclass[letterpaper,12pt,titlepage,oneside,final]{book}
 
% For PDF, suitable for double-sided printing, change the PrintVersion variable below
% to "true" and use this \documentclass line instead of the one above:
\documentclass[letterpaper,12pt,titlepage,openright,twoside,final]{book}


% This package allows if-then-else control structures.
\usepackage{ifthen}
\newboolean{PrintVersion}
\setboolean{PrintVersion}{false} 
% \setboolean{PrintVersion}{true} 
% CHANGE THIS VALUE TO "true" as necessary, to improve printed results 
% for hard copies by overriding some options of the hyperref package.

%%%%%%%%%%%%%%%%%%%%%
% MATLAB Code
\usepackage[framed,numbered,autolinebreaks,useliterate]{mcode}
%%%%%%%%%%%%%%%%%%%%%

%%%%%%%%%%%%%%%%%%%%%
% Algorithm 
\usepackage[english,algo2e,algoruled,vlined,linesnumbered]{algorithm2e}
%%%%%%%%%%%%%%%%%%%%%

%%%%%%%%%%%%%%%%%%%%%
% Table
\usepackage{booktabs}
%%%%%%%%%%%%%%%%%%%%%

%%%%%%%%%%%%%%%%%%%%%
% Enable Subfigures
\usepackage{subfigure}
%%%%%%%%%%%%%%%%%%%%%

%%%%%%%%%%%%%%%%%%%%%
% Enable todonotes
\usepackage{todonotes}
%%%%%%%%%%%%%%%%%%%%%

% Load your needed packages and other commands of yours.
% Load your needed packages and other commands of yours here:
%\usepackage{} % ... note that old .sty files can be included here

















%--------------------------------------------------------------------------
% Do NOT edit the rest of the preample UNLESS YOU KNOW WHAT YOU'RE DOING!
%--------------------------------------------------------------------------

\ifthenelse{\boolean{PrintVersion}}{
\usepackage[top=1in,bottom=1in,left=0.75in,right=1.25in]{geometry}   % For twoside document
}{
\usepackage[top=1in,bottom=1in,left=0.75in,right=1.25in]{geometry}   % For oneside document
}

\usepackage{amsmath,amssymb,amstext} % Lots of math symbols and environments
\usepackage{graphicx} % For including graphics 

\usepackage{nomentbl} 
\makenomenclature 

\usepackage{ifpdf}

\newcommand{\href}[1]{#1} % does nothing, but defines the command so the
    % print-optimized version will ignore \href tags (redefined by hyperref pkg).
%\newcommand{\texorpdfstring}[2]{#1} % does nothing, but defines the command
% Anything defined here may be redefined by packages added below...


% Hyperlinks make it very easy to navigate an electronic document.
% In addition, this is where you should specify the thesis title
% and author as they appear in the properties of the PDF document.
% Use the "hyperref" package 
% N.B. HYPERREF MUST BE THE LAST PACKAGE LOADED; ADD ADDITIONAL PKGS ABOVE
\usepackage[\ifpdf pdftex,\fi letterpaper=true,pagebackref=false]{hyperref} % with basic options
		% N.B. pagebackref=true provides links back from the References to the body text. This can cause trouble for printing.
\hypersetup{
    plainpages=false,       % needed if Roman numbers in frontpages
    pdfpagelabels=true,     % adds page number as label in Acrobat's page count
    bookmarks=true,         % show bookmarks bar?
    unicode=false,          % non-Latin characters in Acrobat's bookmarks
    pdftoolbar=true,        % show Acrobat's toolbar?
    pdfmenubar=true,        % show Acrobat's menu?
    pdffitwindow=false,     % window fit to page when opened
    pdfstartview={FitH},    % fits the width of the page to the window
%    pdftitle={uOttawa\ LaTeX\ Thesis\ Template},    % title: CHANGE THIS TEXT!
%    pdfauthor={Author},    % author: CHANGE THIS TEXT! and uncomment this line
%    pdfsubject={Subject},  % subject: CHANGE THIS TEXT! and uncomment this line
%    pdfkeywords={keyword1} {key2} {key3}, % list of keywords, and uncomment this line if desired
    pdfnewwindow=true,      % links in new window
    colorlinks=true,        % false: boxed links; true: colored links
    linkcolor=blue,         % color of internal links
    citecolor=green,        % color of links to bibliography
    filecolor=magenta,      % color of file links
    urlcolor=cyan           % color of external links
}
\ifthenelse{\boolean{PrintVersion}}{   % for improved print quality, change some hyperref options
\hypersetup{	% override some previously defined hyperref options
%    colorlinks,%
    citecolor=black,%
    filecolor=black,%
    linkcolor=black,%
    urlcolor=black}
}{} % end of ifthenelse (no else)

\usepackage{fancyhdr,lastpage} % Change caption style; changes headers and page styles etc.
\usepackage{epstopdf}
\epstopdfsetup{suffix={}}


% This is where thesis margins and spaces are set.
% Setting up the page margins...
% A minimum of 1 inch (72pt) margin at the
% top, bottom, and outside page edges and a 1.125 in. (81pt) gutter
% margin (on binding side). While this is not an issue for electronic
% viewing, a PDF may be printed, and so we have the same page layout for
% both printed and electronic versions, we leave the gutter margin in.
% Set margins:
\setlength{\marginparwidth}{0pt} % width of margin notes
% N.B. If margin notes are used, you must adjust \textwidth, \marginparwidth
% and \marginparsep so that the space left between the margin notes and page
% edge is less than 15 mm (0.6 in.)
\setlength{\marginparsep}{0pt} % width of space between body text and margin notes
\setlength{\evensidemargin}{0.125in} % Adds 1/8 in. to binding side of all 
% even-numbered pages when the "twoside" printing option is selected
\setlength{\oddsidemargin}{0.125in} % Adds 1/8 in. to the left of all pages
% when "oneside" printing is selected, and to the left of all odd-numbered
% pages when "twoside" printing is selected
\setlength{\textwidth}{6.375in} % assuming US letter paper (8.5 in. x 11 in.) and 
% side margins as above
\raggedbottom

% The following statement specifies the amount of space between
% paragraphs. Other reasonable specifications are \bigskipamount and \smallskipamount.
\setlength{\parskip}{\medskipamount}

% The following statement controls the line spacing.  The default
% spacing corresponds to good typographic conventions and only slight
% changes (e.g., perhaps "1.2"), if any, should be made.
\renewcommand{\baselinestretch}{1} % this is the default line space setting

% By default, each chapter will start on a recto (right-hand side)
% page.  We also force each section of the front pages to start on 
% a recto page by inserting \cleardoublepage commands.
% In many cases, this will require that the verso page be
% blank and, while it should be counted, a page number should not be
% printed.  The following statements ensure a page number is not
% printed on an otherwise blank verso page.
\let\origdoublepage\cleardoublepage
\newcommand{\clearemptydoublepage}{%
  \clearpage{\pagestyle{empty}\origdoublepage}}
\let\cleardoublepage\clearemptydoublepage



\fancypagestyle{myFancy}{%
  \fancyhf{}% Clear header and footer
  \fancyhead[LE,RO]{\bfseries\nouppercase{\rightmark}}
  \fancyhead[LO,RE]{\bfseries\nouppercase{\leftmark}}
  \fancyfoot[R]{Page \thepage\ of \pageref{LastPage}}% Custom footer
  \fancyfoot[L]{G.~Janiak \& K.~Vonckx (\nameOfUniversity)}% Custom footer
  \renewcommand{\headrulewidth}{0.4pt}% Line at the header visible
  \renewcommand{\footrulewidth}{0.1pt}% Line at the footer visible
}


%======================================================================
%   L O G I C A L    D O C U M E N T -- the content of your thesis
%======================================================================
\begin{document}

% For a large document, it is a good idea to divide your thesis
% into several files, each one containing one chapter.
% To illustrate this idea, the "front pages" (i.e., title page,
% declaration, borrowers' page, abstract, acknowledgements,
% dedication, table of contents, list of tables, list of figures,
% nomenclature).
%----------------------------------------------------------------------
% FRONT MATERIAL
%----------------------------------------------------------------------
%
% C O V E R  P A G E
% ------------------
\newcommand{\thesisauthor}{Glenn Janiak and Ken Vonckx}
\newcommand{\advisor}{Dr. Suruz Miah}
\newcommand{\thesistitlecoverpage}{%
Smart Control of 2-Degree of Freedom Helicopters 
}
%\newcommand{\degree}{Ph.D.} % possible values are:
                            % M.A. / M.A.Sc. / M.Sc. / MCS / Ph.D.
\newcommand{\nameofprogram}{Electrical and Computer Engineering Department}
\newcommand{\academicunit}{Caterpillar College of Engineering and Technology}
%\newcommand{\faculty}{Faculty of Engineering}
\newcommand{\nameOfUniversity}{Bradley University}
\newcommand{\graduationyear}{2019}
%
% T I T L E   P A G E
% -------------------
% Last updated May 24, 2011, by Stephen Carr, IST-Client Services
% The title page is counted as page `i' but we need to suppress the
% page number.  We also don't want any headers or footers.
\pagestyle{empty}
\pagenumbering{roman}

% The contents of the title page are specified in the "titlepage"
% environment.
\begin{titlepage}
        \begin{center}
        \vspace*{1.0cm}

        \Huge
        {\bf \thesistitlecoverpage }

        \vspace*{1.0cm}

        \normalsize
        by \\

        \vspace*{1.0cm}

        \Large
        \thesisauthor\\
        Advisor:~\advisor\\

        \vspace*{3.0cm}

        % \normalsize
        % Thesis submitted to the\\
        % Faculty of Graduate and Postdoctoral Studies\\
        % In partial fulfillment of the requirements\\
        % For the \degree~degree in\\
        % \nameofprogram\\

        \vspace*{2.0cm}

        \nameofprogram\\
        \academicunit\\
        %\faculty\\
        \nameOfUniversity\\

        \vspace*{4.0cm}

        \copyright~\thesisauthor, Peoria, Illinois, \graduationyear\\
        \end{center}
\end{titlepage}

% The rest of the front pages should contain no headers and be numbered using Roman numerals starting with `ii'
% PRELIMINARY PAGES

\pagestyle{plain}
\setcounter{page}{2}

\cleardoublepage % Ends the current page and causes all figures and tables that have so far appeared in the input to be printed.
% In a two-sided printing style, it also makes the next page a right-hand (odd-numbered) page, producing a blank page if necessary.



%%% Local Variables:
%%% mode: latex
%%% TeX-master: "../thesisMain"
%%% End:




%
% R E S T  O F  F R O N T  P A G E S
% ----------------------------------
% % D E C L A R A T I O N   P A G E
% -------------------------------
  % This page is not needed for a uOttawa thesis. Don't include it.
  % It is designed for an electronic thesis.
  \noindent
I hereby declare that I am the sole author of this thesis. This is a true copy of the thesis, including any required final revisions, as accepted by my examiners.

  \bigskip
  
  \noindent
I understand that my thesis may be made electronically available to the public.

\cleardoublepage
%\newpage
 %This is not needed in a uOttawa thesis.
%
% Edit the following 3 files with your abstract, acknowledgements, 
% and dedication.
% A B S T R A C T
% ---------------

\begin{center}\textbf{Abstract}\end{center}

%This paper proposes a strategy for testing and comparing three control algorithms, LQR, LQG, and ADP, to control two two-DOF helicopters from a mobile device.  We will be using Raspberry Pi 3's as terminals for the wireless communication and MATLAB as our primary coding language.

Smart building energy management and/or monitoring has recently emerged as a
promising task due to global climate change issues. Such a task has received thorough attention
due to the advent of emerging Internet-of-Things (IoT) technology. Numerous building energy
management systems are commercially available in the market. Most of them lack modularity
or are driven by an overwhelming degree of manual configurations for performing building
automation. In this project, we develop a modular IoT-based framework for a smart building
energy management system (BEMS). The proposed framework is open-source and leverages the
features of decentralized multi-agent systems. The principal component of the framework is the
BEMS core that implements building automation features, such as an optimal scheduler,
displaying weather information, detecting new IoT devices, monitoring energy consumption,
and remotely controlling operation of IoT devices.
\cleardoublepage
%\newpage


%%% Local Variables:
%%% mode: latex
%%% TeX-master: "../finalReportMainV1"
%%% End:

% A C K N O W L E D G E M E N T S
% -------------------------------

\begin{center}\textbf{Acknowledgements}\end{center}

    \par Thanks to Dr. Suruz Miah for his advisement and preliminary work on this project.
    \par Thanks to Mr. Christopher Mattus for his assistance in setting up our lab environment.
    \par Thank you to everyone else who helped make this project possible and successful.


\cleardoublepage
%\newpage



%%% Local Variables:
%%% mode: latex
%%% TeX-master: "../finalReportMainV1"
%%% End:

% D E D I C A T I O N
% -------------------

\begin{center}\textbf{Dedication}\end{center}

This is dedicated to our loved ones and friends who supported us through our time working on this project.

\cleardoublepage
%\newpage


%%% Local Variables:
%%% mode: latex
%%% TeX-master: "../finalReportMainV1"
%%% End:

%
%
% No need to edit this file.
% T A B L E   O F   C O N T E N T S
% ---------------------------------
\renewcommand\contentsname{Table of Contents}
\tableofcontents
\cleardoublepage
\phantomsection
%\newpage

% L I S T   O F   T A B L E S
% ---------------------------
\addcontentsline{toc}{chapter}{List of Tables}
\listoftables
\cleardoublepage
\phantomsection		% allows hyperref to link to the correct page
%\newpage

% L I S T   O F   F I G U R E S
% -----------------------------
\addcontentsline{toc}{chapter}{List of Figures}
\listoffigures
\cleardoublepage
\phantomsection		% allows hyperref to link to the correct page
%\newpage


%
% No need to edit this file. But you may want to comment the whole line if you
% don't have or want a Nomenclature section.
% L I S T   O F   S Y M B O L S
% -----------------------------
% To include a Nomenclature section
\addcontentsline{toc}{chapter}{\textbf{Nomenclature}}

\renewcommand{\nomname}{Nomenclature}
\renewcommand{\nomAname}{\textbf{\large Abbreviations}}
\renewcommand{\nomGname}{\textbf{\large Mathematical Symbols}}
\renewcommand{\nomXname}{\textbf{\large Superscripts}}
\renewcommand{\nomZname}{\textbf{\large Subscripts}}

\printnomenclature
\cleardoublepage
\phantomsection % allows hyperref to link to the correct page
% \newpage


\nomAname
\bigbreak
\begin{itemize}
    \item[]\textbf{2-DOF} - 2~Degrees-Of-Freedom
    \item[]\textbf{ADP} - Approximate Dynamic Programming
    \item[] \textbf{LQG} - Linear Quadratic Gaussian
    \item[]\textbf{LQR} - Linear Quadratic Regulator
    \item[]\textbf{RMSE} - Root Mean Square Error
    \item[]\textbf{SPI} - Serial Peripheral Interface
    \item[]\textbf{TCP} - Transmission Control Protocol
    \item[]\textbf{UDP} - User Datagram Protocol
    \item[]\textbf{USB} - Universal Serial Bus
\end{itemize}

\nomGname
\bigbreak
\begin{itemize}
    \item[]$D_p$ - viscous damping of the pitch axis
    \item[]$D_y$ - viscous damping of the yaw axis
    \item[]$J_p$ - moment of inertia about pitch axis
    \item[]$J_y$ - moment of inertia about yaw axis
    \item[]$K_{sp}$ - stiffness of the axes
    \item[]$K_{pp}$ - pitch motor thrust constant
    \item[]$K_{py}$ - thrust constant acting on the pitch angle from the yaw motor
    \item[]$K_{yp}$ - thrust constant acting on yaw angle from pitch motor
    \item[]$K_{yy}$ - yaw motor thrust constant
\end{itemize}


%%% Local Variables: 
%%% mode: latex
%%% TeX-master: "../uottawa-thesis"
%%% End:   


% Change page numbering back to Arabic numerals
\pagenumbering{arabic}



%

% Redefine the plain page style
\fancypagestyle{plain}{%
  \fancyhf{}%
  \fancyfoot[R]{Page \thepage\ of \pageref{LastPage}}%
  \fancyfoot[L]{G.~Janiak \& K.~Vonckx (\nameOfUniversity)}%  
  \renewcommand{\headrulewidth}{0pt}% Line at the header invisible
  \renewcommand{\footrulewidth}{0.1pt}% Line at the footer visible
}
\pagestyle{myFancy}


%----------------------------------------------------------------------
% MAIN BODY
%---------------------------------------------------------------------- 
% Chapters 
% Include your "sub" source files here (must have extension .tex)
%======================================================================
\chapter{Introduction}
\label{ch: Chapter1}
%======================================================================

\section{Project Description}
The Internet of Things is a large network of embedded devices
\replace{like}{, such as} sensors,
wearables, and appliances capable of receiving control commands and reporting
data over the Internet. Many industries take advantage of this infrastructure to
improve process flow and efficiency. One of these industries is the commercial
and industrial sector which often employ building automation or management
systems to help better manage processes or devices in a building like air
conditioning systems, lights or industrial machines. Technological advancements
have allowed devices to become connected to a building network through
communication protocols like WiFi, Ethernet, Modbus, or BACnet which enables
them to be more easily controlled from a central system.


Our overall goal is to create a platform in which users can login and access devices
connected to a building's energy supply. This will allow the user to closely
monitor energy usage throughout a commercial or residential building.
Additionally, the user will be able to control devices connected to the
platform, allowing precise control over the building's energy use. The minimum
viable product of this project will be to connect to 1 to 3 devices via the the
platform and control them.

\section{Prior Work}
The previous senior project group worked on creating a remote motor control system with a Raspberry Pi and XBee module. To supply power directly to the motor and XBee with a single power supply, they used a buck-boost converter capable of stepping down power supply voltage to 3.3 V. They implemented a Python script that uses the Linux command \texttt{nmap} to search for the Raspberry Pi on the network. The Raspberry Pi connects to the transmitter XBee via a serial connection to transmit commands to the receiver XBee mounted to the motor. They were able to simply toggle the motor on and off and partially implemented this functionality in BEMOSS (Building Energy Management Open Source Software). A second portion of their project was implementing an HVAC control algorithm capable of controlling the temperature of a home with the Linear Quadratic Regulator control algorithm.

During a research project led by Miah \textit{et al.,} the DC motor interface was successfully
implemented into the BEMOSS platform which required adding a great deal of
\add{source} code
to the platform to fully implement the device as BEMOSS does not support adding
new devices currently. The research also consisted of adding a new
\add{features, such as } logo to the
platform for the motor. A possible improvement our platform will make is helping
developers on the project to add new devices more easily.

One of the improvements our project makes over the previous is the use of UDP
\add{(user datagram protocol)} multicasting which does not require any superuser privileges to locate an
embedded computer (Beaglebone Blue\add{, for example}). This implementation is also more convenient
because it does not require installing any packages as multicasting is natively
supported by most embedded Linux systems. If in the future, multiple embedded
computers are needed to be discovered, the multicasting system works elegantly
as each device can simply listen on a provided multicast group which consists of
an IP address and port number.

\section{Background Study}
A great deal of research has been conducted in the field of building energy management. Multiple different platforms have been developed that demonstrate the vast amount of features and possibilities that exist in developing these types of software.

One such platform is the BEMOSS from the Virginia Polytechnic Institute and State
University Advanced Research Institute. BEMOSS features include being
open-source, allowing for multiple communication technologies, and supporting
many common IoT devices. This system allows for monitoring and controlling a
variety of devices securely\todo[inline]{Need citation here}.

Another open-source platform is rEMpy or residential Energy Management in
Python. The main focus of this platform developed by a team of researchers at
Universit\`{a} Politecnica delle Marche in Ancona, Italy was to simulate the
energy flow of a residential home. With the data collected from simulations, it
was possible for them to use forecasting and prediction algorithms to predict
various quantities \replace{like}{, such as} energy usage and power consumption over a given time period. Their energy management system consists of a well-defined structure with components like a user interface, database, and optimal scheduler that communicates with modules like a prediction and thermal model. In terms of software technologies, it uses the Django web framework which provides a lot of built-in functionality for web development like a user-authentication system and object-relational mapping. Adapting the project to this framework could be a great way to help improve security and optimize database queries~\cite{fagiani2017}.

One of the works in the literature discusses Model Predictive Control (MPC)
which is a modern process control algorithm capable of taking into account
current and previous time values to improve building
automation~\cite{Mayer2017}. A multilevel hierarchy is presented to split
control of a building into two levels: the energy supply level and user level.
Example usage of this control scheme provided are temperature control of a
building and its corresponding zones and interaction with a smart grid. Studies
were performed on a commercial building with the building's energy supply system
and hierarchy model predictive controller implemented in Matlab.

In~\cite{8246800}, \remove{the }authors describe similar functional requirements to our
own BEMS Core. They periodically gather power consumption and device status data
and send that data to a database. We are doing exactly what they describe using
the Apache Cassandra database. The authors also use an "analytics [sic] engine
to process [the data] and generate reports, graphs, and charts." Our system only
generates a power consumption plot for a given day, but we are still meeting
this same requirement - just on a smaller scale. The communication with devices
and power consumption reports are to be accessible through a web-based
application that is easy to use. Again, our implementation is simple and very
intuitive for the end-user. Finally, these authors mention a user-login feature.
They explain that their system would include multiple levels of accessibility
based on the user privileges. Although our system does not currently have a
user-login feature, it simply uses a router to access devices that are also
connected to the router. Since this is a closed system, users need only protect
their router with a secure password.

A control scheme in~\cite{Barchi2018} is presented to manage a photovoltaic
array and battery energy storage system. Tests were conducted in a shopping mall
with an electronic load to emulate the power consumption of the building.
Potentially, a similar technique could be used to model the energy demand of a
house in Simulink or Simscape model. Through their scheme, an intelligent BEMS
is used to collect measurable data \replace{like}{, such as} power, voltage, and current and stored for offline analysis later. Their platform collects data every five minutes compared to the 20 second polling rate currently configured in our platform. A control algorithm was presented proving that grid energy is consumed more heavily during periods of low power demand rather than PV power.

% \section{Problem Statement}
% Despite several open-source building energy management platforms, such as BEMOSS, available in the market, the need for an effective and efficient open-source building energy management platform that takes into account the planning and technical challenges in integrating different load controllers
% with the microgrid as well as the smart grids is quite prominent. The proposed energy management platform is expected to address the huge energy demand in residential and commercial buildings despite limited number of sources of energy by incorporating learning, control, and estimation strategies as discussed in the following section.

% A high level architecture of the proposed building energy management system integrating different components of power systems and smart devices is shown in Figure 2. A microgrid is connected to the main grid through a Point of Common Coupling (PCC) to enable or disable islanded mode operation. Various renewable DERs are connected within the microgrid which have the capability of either supplying energy directly to the main grid or storing energy in energy storage devices such as batteries or supercapacitors. The core system will facilitate communication between many different building systems and electrical devices within a building. Examples of different systems include lighting systems and HVAC systems. Smart meters can enable two-way communication between an energy supplier and the BEMS. Various IoT appliances and devices in the building including washing machines, refrigerators, smart TVs, dishwashers, and WiFi-enabled PHEV chargers can be configured for communication with the central system to help optimize electrical consumption. Uncontrollable loads like computers and microwaves may be monitored for power consumption statistics and temperature measuring devices can be connected for determining proper HVAC operation. Utility operators, Ameren, for instance, can be utilized for demand response purposes. Lastly, mobile devices including smart phones and tablets can be used to remotely monitor data from the BEMS core. Usually, a building energy management system is primarily aimed in minimizing energy consumption through building automation strategies and the usage of energy management policies
% and algorithms. The building energy management platform, BEMOSS, developed by researchers at the Virginia Tech research lab, focuses mainly on incorporating building automation strategies, which is currently being used as a starting point for the proposed BEMS platform. The existing BEMOSS platform, for example, contains features for controlling electronic/electrical devices such as scheduling, power consumption monitoring, device control, and usage monitoring. However, no learning, control, and estimation strategies have been developed to help buildings lower energy costs, address the energy demand and smart integration microgrid and smart grid, which proves to be a potential shortcoming of the current BEMOSS.

The ultimate goal of this research is to design and develop a new building energy management
platform incorporating learning, control, and estimation strategies, which is expected to remedy the
shortcomings of some existing platforms, BEMOSS, for instance.

%----------------------------------------------------------------------
\section{Report Organization}
%----------------------------------------------------------------------
This report is organised into 5 chapters.  Chapter \ref{ch: Chapter1} discusses
what this project hopes to accomplish and what similar projects have completed.
Chapter \ref{ch: Chapter2} explains the architecture. Chapter \ref{ch: Chapter3}
provides a brief explanation of the control algorithms and architectures used.
Chapter \ref{ch: Chapter4} contains the results from \replace{our simulations}{a
few experiments}. Chapter \ref{ch: Chapter5} concludes the \replace{paper}{report} and provides insight into future directions.

%%% Local Variables:
%%% mode: latex
%%% TeX-master: "../finalReport"
%%% End:

% \input{parts/20-modeling2-DOF.tex}
% \input{parts/30-controlAlgorithms.tex}
% \input{parts/40-simulations.tex}
% \input{parts/50-implementation.tex}
% \input{parts/60-conclusion.tex}

% %----------------------------------------------------------------------
% % APPENDICES
% %---------------------------------------------------------------------- 
\appendix
% % Designate with \appendix declaration which just changes numbering style 
% % from here on
% % Add a title page before the appendices and a line in the Table of Contents
\addcontentsline{toc}{chapter}{APPENDICES} 
% %

% \input{parts/A-quanserParam.tex}
% \input{parts/B-simulation.tex}
% %\input{parts/C-USB.tex}
% \input{parts/D-RP.tex}
% \input{parts/E-ADP.tex}
% \input{parts/tutorial.tex}

% %----------------------------------------------------------------------
% % END MATERIAL
% %----------------------------------------------------------------------

% % B I B L I O G R A P H Y
% % -----------------------
% %
% % The following statement selects the style to use for references.  It controls the sort order of the entries in the bibliography and also the formatting for the in-text labels.
\bibliographystyle{plain}
% % This specifies the location of the file containing the bibliographic information.  
% % It assumes you're using BibTeX (if not, why not?).
% \ifthenelse{\boolean{PrintVersion}}{
% \cleardoublepage % This is needed if the book class is used, to place the anchor in the correct page,
%                  % because the bibliography will start on its own page.
% }{
% \clearpage       % Use \clearpage instead if the document class uses the "oneside" argument
% }
% \phantomsection  % With hyperref package, enables hyperlinking from the table of contents to bibliography             
% % The following statement causes the title "References" to be used for the bibliography section:
% % \renewcommand*{\bibname}{References}
% Bibliography 
\renewcommand{\bibname}{Bibliography}

% Add the References to the Table of Contents
\addcontentsline{toc}{chapter}{\textbf{References}}

\bibliography{bib/refsHelicopter,bib/refsSuruzWeb}
% Tip 5: You can create multiple .bib files to organize your references. 
% Just list them all in the \bibliogaphy command, separated by commas (no spaces).


%----------------------------------------------------------------------
\end{document}
%======================================================================



%%% Local Variables: 
%%% mode: latex
%%% TeX-master: t
%%% End: 
